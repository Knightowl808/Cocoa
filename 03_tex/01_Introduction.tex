\chapter{Introduction}
\label{ch:introduction}

Volatility forecasting is fundamental to financial risk management, portfolio
allocation, and derivative pricing \parencite{poonForecastingVolatilityFinancial}.
The literature has developed multiple approaches, including historical rolling
averages, implied volatility from options prices, stochastic volatility models,
and various parametric specifications. Among these, two modeling paradigms have
become dominant. The first treats volatility as a latent process that must be
inferred from returns. The Generalized Autoregressive Conditional Heteroskedasticity
(GARCH) model of \textcite{bollerslev1986generalized} represents the canonical
approach in this paradigm, modeling conditional variance as a function of past
shocks and past variances. The second paradigm, enabled by the availability of
high-frequency data, treats volatility as directly observable through realized
measures computed from intraday returns \parencite{andersen2001distribution}.
Within this paradigm, the Heterogeneous Autoregressive (HAR) model of
\textcite{corsiSimpleApproximateLongMemory2008} has emerged as the dominant
forecasting framework, offering a parsimonious structure that captures the
long-memory properties of volatility while remaining simple to estimate.

The majority of volatility forecasting research focuses on conditional mean
forecasts: the expected level of future volatility given current information.
For many applications, however, the mean provides incomplete guidance. Risk
managers, portfolio allocators, and hedgers require information about the
tails of the volatility distribution---specifically, bounds that volatility
is unlikely to exceed. Value-at-Risk calculations, stress testing, and margin
requirements all depend on quantile estimates rather than mean forecasts
\parencite{christoffersen1998evaluating}. Quantile regression, introduced by
\textcite{koenker1978regression}, provides a natural framework for estimating
conditional quantiles directly, without distributional assumptions.

Recent work has combined these approaches. \textcite{haugomParsimoniousQuantileRegression2016}
introduced HAR-QREG, applying quantile regression to the HAR framework for
direct Value-at-Risk estimation. Subsequent studies have extended this approach
to equity markets \parencite{lyocsaImprovingStockMarket2021, huangHighfrequencyApproachVaR2022}
and precious metals \parencite{songGoldFuturesVolatility2025}. However,
agricultural commodities remain understudied. While HAR models have been applied
to agricultural futures \parencite{degiannakisForecastingRealizedVolatility2022,
alfeusForecastingVolatilityCommodity2022}, no study applies HAR-based quantile
regression to this asset class.

Cocoa futures present a particularly relevant case. The cocoa market is
characterized by high geographic concentration of production, vulnerability
to weather and disease shocks, and pronounced price volatility
\parencite{rognaAnalysisCocoaMarket2025}. Between 2022 and 2024, international
cocoa prices approximately quadrupled - an increase unprecedented in the
commodity's modern trading history. For chocolate manufacturers, who
commit to purchases months in advance through futures contracts,
accurate volatility forecasts at long horizons are essential for hedging
decisions and procurement planning. Yet the volatility forecasting literature
provides limited guidance at such horizons: existing HAR-QR studies extend
only to approximately one month \parencite{lyocsaImprovingStockMarket2021},
and \textcite{poonForecastingVolatilityFinancial} conclude that forecast
accuracy degrades substantially beyond six months.

At long forecast horizons, the autoregressive signal from lagged volatility
fades, motivating the search for additional predictors. Climate variables
offer a potential source of information that operates on precisely these
longer timescales. \textcite{bouriNinoForecastabilityOilprice2021} demonstrate
that the El Ni\~no--Southern Oscillation (ENSO) index improves oil volatility
forecasts at horizons of two to four years. \textcite{bonatoNinoNinaForecastability2023}
extend this finding to agricultural commodities, documenting that ENSO predictive
value strengthens at longer horizons. For cocoa, whose production is concentrated
in West Africa and sensitive to rainfall patterns affected by ENSO phases
\parencite{ubilavaRoleNinoSouthern2018}, climate augmentation may prove
particularly valuable.

This thesis addresses the following research question:
\begin{quote}
Can HAR-based quantile regression models forecast cocoa futures volatility at
horizons relevant for procurement and hedging (1 month to 12 months), and does
the ENSO climate index improve tail-risk forecasts at long horizons?
\end{quote}

We address this question by estimating HAR-QR models at multiple quantiles
($\tau \in \{0.05, 0.25, 0.50, 0.75, 0.95\}$) and horizons (1 day, 1 week,
1 month, 3 months, 6 months, 12 months). At each horizon, we compare
out-of-sample forecast accuracy against standard benchmarks: historical
rolling volatility, GARCH(1,1), and HAR estimated by ordinary least squares.
For long horizons, we test whether adding ENSO improves forecasts, with
particular focus on the upper quantile ($\tau = 0.95$) relevant for risk
management applications.

The thesis makes three contributions to the literature. First, it applies
HAR-QR to an agricultural commodity. While HAR models have been applied to
cocoa within broad commodity surveys \parencite{alfeusForecastingVolatilityCommodity2022,
bonatoForecastingRealizedVolatility2024}, and HAR-QR has been applied to
equities and precious metals, no study has combined these approaches for
agricultural commodities.

Second, the thesis extends HAR-QR to horizons beyond existing applications.
The literature currently provides HAR-QR evidence only for horizons up to
approximately one month \parencite{lyocsaImprovingStockMarket2021}. This
research tests whether quantile forecasts remain useful at horizons of 6 to
12 months, where industrial hedgers require guidance but the literature
offers none.

Third, the thesis incorporates climate variables into a quantile regression
framework for commodity volatility. While \textcite{bouriNinoForecastabilityOilprice2021}
demonstrate that ENSO improves mean volatility forecasts for oil, no study
has tested whether ENSO enhances quantile forecasts for agricultural commodities.
This extension is particularly relevant for cocoa, given the documented
sensitivity of tropical agricultural production to ENSO phases.

The remainder of the thesis is organized as follows.
Chapter~\ref{ch:literature} reviews the literature on volatility measurement,
HAR models, quantile regression in finance, and climate-commodity linkages,
establishing the research gap this thesis addresses.
Chapter~\ref{ch:methodology} specifies the HAR-QR model, the multi-horizon
projection approach, and the evaluation framework.
Chapter~\ref{ch:data} describes the ICE London cocoa futures data and validates
the daily volatility proxy against realized volatility from intraday data.
Chapter~\ref{ch:results} presents empirical results organized by forecast
horizon.
Chapter~\ref{ch:discussion} interprets findings in the context of risk
management and procurement applications.
Chapter~\ref{ch:conclusion} summarizes key results and directions for future
research.
